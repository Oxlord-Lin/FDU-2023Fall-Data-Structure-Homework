\documentclass{article}
\usepackage[UTF8]{ctex}
% Replace `letterpaper' with`a4paper' for UK/EU standard size
\usepackage[a4paper,top=2cm,bottom=2cm,left=3cm,right=3cm,marginparwidth=1.75cm]{geometry}

% Useful packages
\usepackage{amsmath}
\usepackage{graphicx}
\usepackage[colorlinks=true, allcolors=blue]{hyperref}
\usepackage{graphicx} %插入图片的宏包
\usepackage{float} %设置图片浮动位置的宏包
\usepackage{subfigure} %插入多图时用子图显示的宏包
\usepackage{parskip}
\usepackage{indentfirst} 
\setlength{\parindent}{2em}
\usepackage{hyperref}  
\usepackage{tikz}
\allowdisplaybreaks
\usepackage{multirow}
\usepackage{amsmath}
\usepackage{amsfonts,amssymb} 
\usepackage{xcolor} % 用于显示颜色
\usepackage{listings} % 用于插入代码
\lstset{
	basicstyle          =   \sffamily,          % 基本代码风格
	keywordstyle        =   \bfseries,          % 关键字风格
	commentstyle        =   \rmfamily\itshape,  % 注释的风格,斜体
	stringstyle         =   \ttfamily,  % 字符串风格
	flexiblecolumns,                % 别问为什么,加上这个
	numbers             =   left,   % 行号的位置在左边
	showspaces          =   false,  % 是否显示空格,显示了有点乱,所以不现实了
	numberstyle         =   \zihao{-5}\ttfamily,    % 行号的样式,小五号,tt等宽字体
	showstringspaces    =   false,
	captionpos          =   t,      % 这段代码的名字所呈现的位置,t指的是top上面
	frame               =   lrtb,   % 显示边框
}

\lstdefinestyle{Python}{
	language        =   Python, % 语言选Python
	basicstyle      =   \zihao{-5}\ttfamily,
	numberstyle     =   \zihao{-5}\ttfamily,
	keywordstyle    =   \color{blue},
	keywordstyle    =   [2] \color{teal},
	stringstyle     =   \color{magenta},
	commentstyle    =   \color{red}\ttfamily,
	breaklines      =   true,   % 自动换行,建议不要写太长的行
	columns         =   fixed,  % 如果不加这一句,字间距就不固定,很丑,必须加
	basewidth       =   0.5em,
}

\title{数据结构第3次上机实验报告}
\author{林子开}

\begin{document}
	\maketitle
	\tableofcontents

\section{中序表达式到后序表达式的转换}
\subsection{python代码}

\lstinputlisting[style = Python,
caption={Postfix evaluator, python源码},
label = {in2post}
]{infix_to_postfix.py}

\subsection{中序表达式到后序表达式的实验结果}

\begin{enumerate}
	\item 
Infix:( A + B ) * C

Postfix:A B + C *

\item
Infix:A + ( B - C )

Postfix:A B C - +

\item
Infix:A * ( B + C ) / D

Postfix:A B C + * D /

\item
Infix:( A + B ) * ( C - D )

Postfix:A B + C D - *

\item 
Infix: A + B * C - D / E

Postfix:A B C * + D E / -

\item
Infix:( A * B ) + ( C / D ) - E

Postfix:A B * C D / + E -

\item
Infix:( A + B ) / ( C + D ) * E

Postfix:A B + C D + / E *

\item
Infix: A * ( B + C ) - ( D * E )

Postfix:A B C + * D E * -

\item
Infix:( A + B ) * ( C - D ) / ( E + F )

Postfix:A B + C D - * E F + /

\item
Infix: A * ( B + ( C * ( D - ( E / ( F + ( G * H ) ) ) ) ) )  / I

Postfix:A B C D E F G H * + / - * + * I /

\end{enumerate}


\section{后序表达式运算器}
\subsection{python代码}
\lstinputlisting[style = Python,
caption={infix to postfix conversion, python源码},
label = {in2post}
]{postfix_calculator.py}

\subsection{后序表达式运算器的实验结果}
\begin{enumerate}
	\item
Infix: 3 5 + 2 7 * /

Result: 0
\item
Infix: 25 5 + 3 * 21 7 / 1 + -

Result: 86
\item
Infix:  5 1 2 + 4 * + 3 - 7 4 5 - + +

Result: 20
\item
Infix:  36 3 / 5 + 2 * 14 - 3 * 24 2 / 1 - + 2 /

Result: 35
\item
Infix:  20 4 - 2 * 14 + 7 / 1 - 5 * 9 + 12 3 / 2 + -

Result: 28
\item
Infix:  10 5 + 2 * 8 - 4 / 3 + 6 * 12 2 * 4 + - 18 3 / 2 * + 5 -

Result: 27
\item
Infix: 24 3 / 6 + 2 * 14 - 2 / 5 + 4 * 16 2 * 3 + - 21 3 / 2 * +

Result: 27
\item
Infix:  20 6 + 2 * 14 - 7 / 1 + 4 * 10 2 * 3 + - 27 3 / 2 * + 4 -

Result: 15
\item
Infix:  8 4 + 3 * 18 - 2 / 7 + 5 * 10 - 2 + 4 * 12 - 6 + 2 * 3 - 2 /

Result: 280
\item
Infix:  36 4 / 7 + 2 * 14 - 2 / 6 + 3 * 12 - 5 + 4 * 16 - 8 + 2 / 5 + 2 3 * - 7 1 / +

Result: 78

\end{enumerate}

\end{document}