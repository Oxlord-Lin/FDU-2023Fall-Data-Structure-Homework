\documentclass{article}
\usepackage[UTF8]{ctex}
% Replace `letterpaper' with`a4paper' for UK/EU standard size
\usepackage[a4paper,top=2cm,bottom=2cm,left=3cm,right=3cm,marginparwidth=1.75cm]{geometry}

% Useful packages
\usepackage{amsmath}
\usepackage{mathrsfs,amsmath}
\usepackage{graphicx}
\usepackage[colorlinks=true, allcolors=blue]{hyperref}
\usepackage{graphicx} %插入图片的宏包
\usepackage{float} %设置图片浮动位置的宏包
\usepackage{subfigure} %插入多图时用子图显示的宏包
\usepackage{parskip}
\usepackage{indentfirst} 
\setlength{\parindent}{2em}
\usepackage{hyperref}  
\usepackage{tikz}
\allowdisplaybreaks
\usepackage{multirow}
\usepackage{amsmath}
\usepackage{amsfonts,amssymb} 
\usepackage{xcolor} % 用于显示颜色
\usepackage{listings} % 用于插入代码
\lstset{
	basicstyle          =   \sffamily,          % 基本代码风格
	keywordstyle        =   \bfseries,          % 关键字风格
	commentstyle        =   \rmfamily\itshape,  % 注释的风格,斜体
	stringstyle         =   \ttfamily,  % 字符串风格
	flexiblecolumns,                % 别问为什么,加上这个
	numbers             =   left,   % 行号的位置在左边
	showspaces          =   false,  % 是否显示空格,显示了有点乱,所以不现实了
	numberstyle         =   \zihao{-5}\ttfamily,    % 行号的样式,小五号,tt等宽字体
	showstringspaces    =   false,
	captionpos          =   t,      % 这段代码的名字所呈现的位置,t指的是top上面
	frame               =   lrtb,   % 显示边框
}

\lstdefinestyle{Python}{
	language        =   Python, % 语言选Python
	basicstyle      =   \zihao{-5}\ttfamily,
	numberstyle     =   \zihao{-5}\ttfamily,
	keywordstyle    =   \color{blue},
	keywordstyle    =   [2] \color{teal},
	stringstyle     =   \color{magenta},
	commentstyle    =   \color{red}\ttfamily,
	breaklines      =   true,   % 自动换行,建议不要写太长的行
	columns         =   fixed,  % 如果不加这一句,字间距就不固定,很丑,必须加
	basewidth       =   0.5em,
}

\title{数据结构 Lab-7 报告}
\author{林子开 21307110161}
\begin{document}
	\maketitle
	\tableofcontents

\section{基于深度搜索的拓扑排序}

\paragraph{算法思路}
对有向无环图进行深度搜索,深度搜索采用递归的方式完成。
记录每个节点的开始探索时间$t_{start}$和结束时间$t_{finish}$,按照结束时间$t_{finish}$
\textbf{逆序排列},则得到各个节点的拓扑排序结果。
该算法已经在文件\texttt{exercise-1.py}文件中实现,敬请参阅。

\paragraph{实验结果}
对助教提供的课程关系图进行拓扑排序,修读的顺序如下(该结果不唯一):

Discrete Mathematics$\;\rightarrow\;$Calculus$\;\rightarrow\;$
Probability and Statistics$\;\rightarrow\;$Computer Systems$\;\rightarrow\;$
Computer Network$\;\rightarrow\;$Computer Architecture$\;\rightarrow\;$
Database$\;\rightarrow\;$Java or C+$\;\rightarrow\;$Data Structure and Algorithm$\;\rightarrow\;$
Object Oriented Programming$\;\rightarrow\;$Software Engineering$\;\rightarrow\;$
Intelligent Systems$\;\rightarrow\;$Project Management$\;\rightarrow\;$Web Application$\;\rightarrow\;$
ALL COURSES$\;\rightarrow\;$Internship$\;\rightarrow\;$Thesis

\section{利用深度搜索解决过河问题}
\paragraph{建立有向图模型}
用一个四元组$[w,g,c,f]$表示狼、羊、菜、人的位置,取$0$表示在左岸,取$1$表示在右岸。
以所有四元组作为图的节点。该图有$2^4=16$个节点,但其中存在不安全的节点,即人不在场的情况下,狼吃羊或羊吃菜的节点。
不安全的节点出度为零,即不会指向任何其他节点。
问题的起点为$[0,0,0,0]$,终点是$[1,1,1,1]$。每次要么单独改变人的位置,要么在狼、羊、菜(前提是与人在同一侧)中
选择一个和人一起过河,因此从每个节点出发,最多有四个可能的邻接节点。此外,注意到该图中存在环,我们要求,
输出的所有可能解中\textbf{不能存在环},否则可行解将有无穷多个。

\paragraph{算法描述}
基于栈实现深度优先搜索。每次从栈中弹出一个节点,不妨记为$N$,先检查节点$N$是否为终点,如果是终点,则打印路径。
如果不是终点,找出节点$N$的\textbf{所有安全的}邻接节点,记为集合$\{adj_k(N)\}$。对集合中的元素$adj_k(N)$依次检查,
若从起点开始到$adj_k(N)$的路径\textbf{出现了环},则忽略$adj_k(N)$;\textbf{若没有出现环},
则将$adj_k(N)$以及从起点到$adj_k(N)$的路径一起压入栈中,这样可以便于之后检查路径中是否出现环。
简而言之,\textbf{每个压入栈中的节点,都是既安全、而且不会造成路径出现环的节点。}
该算法已经在文件\texttt{exercise-2.py}文件中实现,敬请参阅。

\paragraph{实验结果}
对过河问题进行求解,得到以下两种解法:

[0, 0, 0, 0]$\;\rightarrow\;$[0, 1, 0, 1]$\;\rightarrow\;$[0, 1, 0, 0]$\;\rightarrow\;$[0, 1, 1, 1]$\;\rightarrow\;$[0, 0, 1, 0]$\;\rightarrow\;$[1, 0, 1, 1]$\;\rightarrow\;$[1, 0, 
1, 0]$\;\rightarrow\;$[1, 1, 1, 1]。 
对该解法进行文字说明:人先把羊带到右岸,人独自回到左岸,人把菜带到右岸,人把羊带回左岸,人把狼带到右岸,人独自回到左岸,人把羊带到右岸。

[0, 0, 0, 0]$\;\rightarrow\;$[0, 1, 0, 1]$\;\rightarrow\;$[0, 1, 0, 0]$\;\rightarrow\;$[1, 1, 0, 1]$\;\rightarrow\;$[1, 0, 0, 0]$\;\rightarrow\;$[1, 0, 1, 1]$\;\rightarrow\;$[1, 0, 
1, 0]$\;\rightarrow\;$[1, 1, 1, 1]。 
对该解法进行文字说明:人先把羊带到右岸,人独自回到左岸,人把狼带到右岸,人把羊带回左岸,人把菜带到右岸,人独自回到左岸,人把羊带到右岸。
\end{document}

% \begin{figure}[H]
% 	\centering
% 	{\includegraphics[width=0.35\textwidth]{image//ignorance.png}} 
% 	\caption{} \label{} 
% \end{figure}


% \lstinputlisting[style = Python,
% caption={Python codes},
% label = {efficient},
% linerange={110-125}]{exercise3.py} 


% \begin{figure}[H]
%     \centering
%     \subfigure[patch size = 11]
%     {\label{} \includegraphics[width=0.49\textwidth]{image//local equalization with patch size = 11.jpg}}
%     \,    
%     \subfigure[patch size = 51]
%     {\label{} \includegraphics[width=0.49\textwidth]{image//local equalization with patch size = 51.jpg}}
%     \,
%     \subfigure[patch size = 151]
%     {\label{} \includegraphics[width=0.49\textwidth]{image//local equalization with patch size = 151.jpg}}
%     \,    
%     \subfigure[patch size = 201]
%     {\label{} \includegraphics[width=0.49\textwidth]{image//local equalization with patch size = 201.jpg}}
%     \caption{local equalization with different patch sizes}\label{} 
% \end{figure}
